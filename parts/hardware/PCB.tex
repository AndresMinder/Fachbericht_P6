\subsection{PCB}
\label{subsec:PCB}
Ein PCB (Printed Circuit Board, dt. Leiterplatte) beherbergt die verwendeten Bauteile und verbindet diese elektronisch miteinander. Die verwendeten Bauteile werden auf das PCB gelötet, wobei es SMD (Surface Mount Device) und THT (Trough Hole Technology) zu unterscheiden gilt. Wie der Name schon sagt, werden THT-Bauteile durch das PCB gesteckt und SMD-Bauteile lediglich auf die Oberfläche gelegt, wobei beide vorgesehene Lötstellen haben auf diese sie gelötet werden um elektrischen Kontakt über Leiterbahnen zu erstellen.\\
\todo[inline]{SMD und THT unterschied per Bild aufzeigen}
Das PCB und das dazugehörige Schema werden in einem EDA-Programm erstellt. Für die Wetterstation wurde das Programm EAGLE (Einfach Anzuwendender Grafischer Layout-Editor) der Firma Autodesk verwendet, wegen der im Namen schon angedeuteten einfachen Anwendung. Das Schema und das dazugehörige PCB-Layout ist im Anhang zu finden. \todo[inline]{Anhang hinzufügen und referenzieren} In den nachfolgenden Kapiteln wird auf das Bestücken der Leiterplatte eingegangen und auf Mängel der ersten Version hingewiesen, sowie die Art und Weise wie diese Mängel behoben wurden.
\subsubsection{Das Bestücken der Leiterplatte}
\todo[inline]{Schreib mal du faule Sau!}
\subsubsection{Mängel der Erstversion}
\todo[inline]{Schreib mal du faule Sau!}
\subsubsection{Verbesserungen für die Zweitversion}
\todo[inline]{Schreib mal du faule Sau!}

